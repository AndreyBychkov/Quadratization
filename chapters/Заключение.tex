% !TEX root = ../Диплом.tex

\section{Заключение}

Существующие алгоритмы полиномиализации и квадратизации не были в достаточной степени формальными, чтобы исполняться на вычислительных устройствах. В данной работе мы предложили формализацию данных алгоритмов на языке графов и свели задачу об оптимальной квадратизации к задаче поиска на графе.

Было показано, какие алгоритмы неинформированного поиска могут справиться с задачей лучше всего. После этого лучший алгоритм был модифицирован до своей эвристической версии. К полученному эвристическому алгоритму был предложен ряд эвристик, после чего было проведено их сравнение. Помимо этого, была проведено разрежение графа, на котором осуществляется поиск, с помощью знаний о предметной области. Таким образом удалось добиться значительного ускорения скорости работы по сравнению с базовыми алгоритмами.

Таким образом, предложенные алгоритмы полиномиализации и квадратизации могут использованы для достаточно быстрого преобразования небольших систем уравнений. Полученный программный пакет, реализованный на языке Python, находится в свободном распространении \cite{QBee}.

Для того, чтобы работать с крупными системами, в будущем будут проведены:
\begin{enumerate}
    \item Оптимизация представления полиномиальных математических выражений, которые могут быть представлены как список кортежей фиксированного размера. Это позволит значительно снизить объём занимаемой памяти и время копирования системы, на данный момент занимающее треть вычислительных ресурсов.
    \item Параллелизация алгоритма квадратизации.
    \item Добавление новых эвристик и синтез уже существующих.
    \item Использование языка программирования Julia. 
\end{enumerate}
