% !TEX root = ../Диплом.tex

\section{Введение}

Динамические системы являются неотъемлемой частью современной прикладной математики. С помощью динамических систем описываются явления в таких областях как физика \cite{physics-example}, химия \cite{chemistry-example}, биология \cite{biology-example}, экономика \cite{economics-example} и многих других. Математические методы позволяют проводить с ними такие операции как упрощение \cite{MOR-book} и анализировать такие их свойства как устойчивость \cite{Strogatz-book} и достижимость \cite{Scott-reachability}. Многие динамические системы, встречающиеся на практике, описываются с помощью нелинейных дифференциальных уравнений. Анализ таких  систем является молодой областью со множеством открытых вопросов, в отличие от анализа линейных систем \cite{MOR-linear-overview}. Поэтому не удивительно, что многие известные методы для работы с нелинейными системами полагаются на аппрокcимацию нелинейных элементов линейными. К сожалению, такой подход часто ведёт к неудовлетворительным результатам. 
Поэтому большой интерес заслужили подходы, целиком полагающиеся на нелинейную природу систем, которыми они оперируют. 
Отдельно выделим метод понижения порядка моделей QLMOR \cite{Gu-PhD, Kramer-Willcox}, который показывает state-of-the-art результаты, сохраняя при этом динамику исходной системы. 
Данный метод полагается на процесс квадратизации --- приведения систем нелинейных дифференциальных уравнений к системе полиномиальных дифференциальных уравнений степени не более двух. 
Примером квадратизации может послужить следующее преобразование (формальное определение и подробные примеры приведены в главах \ref{sec:definitions}, \ref{sec:polynomialization} и \ref{sec:quadratiaztion}):

\[
     \dot x = \frac{1}{1 + e^x} 
\Longrightarrow
\begin{cases}
    \dot x = y_1 \\
    \dot y_0 = y_0 y_1 \\
    \dot y_1 = -y_1 y_2 \\
    \dot y_2 = y_1 y_2 - 2y_2^2
\end{cases}
\]

На данный момент не существует достаточно эффективных или хотя бы реализованных алгоритмов квадратизации --- она всегда совершается вручную.
Из-за этого, даже для сравнительно небольших систем уравнений проводить её крайне долго и сложно. 
Более того, нет гарантий, что мы полученная квадратизацию имеет наименьшую возможную размерность, что критично в задаче понижения порядка моделей. 

В данной работе мы предлагаем алгоритм квадратизации, описываем его реализацию и модификации, и демонстрируем работу реализации на примерах систем из литературы.